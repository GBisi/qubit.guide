% Options for packages loaded elsewhere
\PassOptionsToPackage{unicode}{hyperref}
\PassOptionsToPackage{hyphens}{url}
%
\documentclass[
]{book}
\usepackage{lmodern}
\usepackage{amsmath}
\usepackage{ifxetex,ifluatex}
\ifnum 0\ifxetex 1\fi\ifluatex 1\fi=0 % if pdftex
  \usepackage[T1]{fontenc}
  \usepackage[utf8]{inputenc}
  \usepackage{textcomp} % provide euro and other symbols
  \usepackage{amssymb}
\else % if luatex or xetex
  \usepackage{unicode-math}
  \defaultfontfeatures{Scale=MatchLowercase}
  \defaultfontfeatures[\rmfamily]{Ligatures=TeX,Scale=1}
\fi
% Use upquote if available, for straight quotes in verbatim environments
\IfFileExists{upquote.sty}{\usepackage{upquote}}{}
\IfFileExists{microtype.sty}{% use microtype if available
  \usepackage[]{microtype}
  \UseMicrotypeSet[protrusion]{basicmath} % disable protrusion for tt fonts
}{}
\makeatletter
\@ifundefined{KOMAClassName}{% if non-KOMA class
  \IfFileExists{parskip.sty}{%
    \usepackage{parskip}
  }{% else
    \setlength{\parindent}{0pt}
    \setlength{\parskip}{6pt plus 2pt minus 1pt}}
}{% if KOMA class
  \KOMAoptions{parskip=half}}
\makeatother
\usepackage{xcolor}
\IfFileExists{xurl.sty}{\usepackage{xurl}}{} % add URL line breaks if available
\IfFileExists{bookmark.sty}{\usepackage{bookmark}}{\usepackage{hyperref}}
\hypersetup{
  pdftitle={Lectures on Quantum Information Science},
  pdfauthor={Artur Ekert},
  hidelinks,
  pdfcreator={LaTeX via pandoc}}
\urlstyle{same} % disable monospaced font for URLs
\usepackage{color}
\usepackage{fancyvrb}
\newcommand{\VerbBar}{|}
\newcommand{\VERB}{\Verb[commandchars=\\\{\}]}
\DefineVerbatimEnvironment{Highlighting}{Verbatim}{commandchars=\\\{\}}
% Add ',fontsize=\small' for more characters per line
\usepackage{framed}
\definecolor{shadecolor}{RGB}{248,248,248}
\newenvironment{Shaded}{\begin{snugshade}}{\end{snugshade}}
\newcommand{\AlertTok}[1]{\textcolor[rgb]{0.94,0.16,0.16}{#1}}
\newcommand{\AnnotationTok}[1]{\textcolor[rgb]{0.56,0.35,0.01}{\textbf{\textit{#1}}}}
\newcommand{\AttributeTok}[1]{\textcolor[rgb]{0.77,0.63,0.00}{#1}}
\newcommand{\BaseNTok}[1]{\textcolor[rgb]{0.00,0.00,0.81}{#1}}
\newcommand{\BuiltInTok}[1]{#1}
\newcommand{\CharTok}[1]{\textcolor[rgb]{0.31,0.60,0.02}{#1}}
\newcommand{\CommentTok}[1]{\textcolor[rgb]{0.56,0.35,0.01}{\textit{#1}}}
\newcommand{\CommentVarTok}[1]{\textcolor[rgb]{0.56,0.35,0.01}{\textbf{\textit{#1}}}}
\newcommand{\ConstantTok}[1]{\textcolor[rgb]{0.00,0.00,0.00}{#1}}
\newcommand{\ControlFlowTok}[1]{\textcolor[rgb]{0.13,0.29,0.53}{\textbf{#1}}}
\newcommand{\DataTypeTok}[1]{\textcolor[rgb]{0.13,0.29,0.53}{#1}}
\newcommand{\DecValTok}[1]{\textcolor[rgb]{0.00,0.00,0.81}{#1}}
\newcommand{\DocumentationTok}[1]{\textcolor[rgb]{0.56,0.35,0.01}{\textbf{\textit{#1}}}}
\newcommand{\ErrorTok}[1]{\textcolor[rgb]{0.64,0.00,0.00}{\textbf{#1}}}
\newcommand{\ExtensionTok}[1]{#1}
\newcommand{\FloatTok}[1]{\textcolor[rgb]{0.00,0.00,0.81}{#1}}
\newcommand{\FunctionTok}[1]{\textcolor[rgb]{0.00,0.00,0.00}{#1}}
\newcommand{\ImportTok}[1]{#1}
\newcommand{\InformationTok}[1]{\textcolor[rgb]{0.56,0.35,0.01}{\textbf{\textit{#1}}}}
\newcommand{\KeywordTok}[1]{\textcolor[rgb]{0.13,0.29,0.53}{\textbf{#1}}}
\newcommand{\NormalTok}[1]{#1}
\newcommand{\OperatorTok}[1]{\textcolor[rgb]{0.81,0.36,0.00}{\textbf{#1}}}
\newcommand{\OtherTok}[1]{\textcolor[rgb]{0.56,0.35,0.01}{#1}}
\newcommand{\PreprocessorTok}[1]{\textcolor[rgb]{0.56,0.35,0.01}{\textit{#1}}}
\newcommand{\RegionMarkerTok}[1]{#1}
\newcommand{\SpecialCharTok}[1]{\textcolor[rgb]{0.00,0.00,0.00}{#1}}
\newcommand{\SpecialStringTok}[1]{\textcolor[rgb]{0.31,0.60,0.02}{#1}}
\newcommand{\StringTok}[1]{\textcolor[rgb]{0.31,0.60,0.02}{#1}}
\newcommand{\VariableTok}[1]{\textcolor[rgb]{0.00,0.00,0.00}{#1}}
\newcommand{\VerbatimStringTok}[1]{\textcolor[rgb]{0.31,0.60,0.02}{#1}}
\newcommand{\WarningTok}[1]{\textcolor[rgb]{0.56,0.35,0.01}{\textbf{\textit{#1}}}}
\usepackage{longtable,booktabs}
\usepackage{calc} % for calculating minipage widths
% Correct order of tables after \paragraph or \subparagraph
\usepackage{etoolbox}
\makeatletter
\patchcmd\longtable{\par}{\if@noskipsec\mbox{}\fi\par}{}{}
\makeatother
% Allow footnotes in longtable head/foot
\IfFileExists{footnotehyper.sty}{\usepackage{footnotehyper}}{\usepackage{footnote}}
\makesavenoteenv{longtable}
\usepackage{graphicx}
\makeatletter
\def\maxwidth{\ifdim\Gin@nat@width>\linewidth\linewidth\else\Gin@nat@width\fi}
\def\maxheight{\ifdim\Gin@nat@height>\textheight\textheight\else\Gin@nat@height\fi}
\makeatother
% Scale images if necessary, so that they will not overflow the page
% margins by default, and it is still possible to overwrite the defaults
% using explicit options in \includegraphics[width, height, ...]{}
\setkeys{Gin}{width=\maxwidth,height=\maxheight,keepaspectratio}
% Set default figure placement to htbp
\makeatletter
\def\fps@figure{htbp}
\makeatother
\setlength{\emergencystretch}{3em} % prevent overfull lines
\providecommand{\tightlist}{%
  \setlength{\itemsep}{0pt}\setlength{\parskip}{0pt}}
\setcounter{secnumdepth}{5}
\usepackage{booktabs}
\ifluatex
  \usepackage{selnolig}  % disable illegal ligatures
\fi
\usepackage[]{natbib}
\bibliographystyle{apalike}

\title{Lectures on Quantum Information Science}
\author{Artur Ekert}
\date{2020-11-14}

\begin{document}
\maketitle

{
\setcounter{tocdepth}{1}
\tableofcontents
}
\hypertarget{prerequisites}{%
\chapter*{Prerequisites}\label{prerequisites}}
\addcontentsline{toc}{chapter}{Prerequisites}

This is a \emph{sample} book written in \textbf{Markdown}. You can use anything that Pandoc's Markdown supports, e.g., a math equation \(a^2 + b^2 = c^2\).

The \textbf{bookdown} package can be installed from CRAN or Github:

\begin{Shaded}
\begin{Highlighting}[]
\FunctionTok{install.packages}\NormalTok{(}\StringTok{"bookdown"}\NormalTok{)}
\CommentTok{\# or the development version}
\CommentTok{\# devtools::install\_github("rstudio/bookdown")}
\end{Highlighting}
\end{Shaded}

Remember each Rmd file contains one and only one chapter, and a chapter is defined by the first-level heading \texttt{\#}.

To compile this example to PDF, you need XeLaTeX. You are recommended to install TinyTeX (which includes XeLaTeX): \url{https://yihui.org/tinytex/}.

\hypertarget{quantum-interference}{%
\chapter{Quantum interference}\label{quantum-interference}}

\begin{quote}
About complex numbers, called probability amplitudes, that, unlike probabilities, can cancel each other out, leading to quantum interference and qualitatively new ways of processing information.
\end{quote}

The classical theory of computation does not usually refer to physics.
Pioneers such as Alan Turing, Alonzo Church, Emil Post and Kurt G"\{o\}del managed to capture the correct classical theory by intuition alone and, as a result, it is often falsely assumed that its foundations are self-evident and purely abstract.
They are not!\footnote{Computation is a physical process. Computation is a physical process. Computation is \ldots{}}

The concepts of information and computation can be properly formulated only in the context of a physical theory --- information is stored, transmitted and processed always by \emph{physical} means.
Computers are physical objects and computation is a physical process.
Indeed, any computation, classical or quantum, can be viewed in terms of physical experiments, which produce \textbf{outputs} that depend on initial preparations called \textbf{inputs}.
Once we abandon the classical view of computation as a purely logical notion independent of the laws of physics it becomes clear that whenever we improve our knowledge about physical reality, we may also gain new means of computation.
Thus, from this perspective, it is not very surprising that the discovery of quantum mechanics in particular has changed our understanding of the nature of computation.
In order to explain what makes quantum computers so different from their classical counterparts, we begin with the rudiments of quantum theory.

\hypertarget{two-basic-rules}{%
\section{Two basic rules}\label{two-basic-rules}}

\hypertarget{quantum-interference-the-failure-of-probability-theory}{%
\section{Quantum interference (the failure of probability theory)}\label{quantum-interference-the-failure-of-probability-theory}}

\hypertarget{superpositions}{%
\section{Superpositions}\label{superpositions}}

\hypertarget{interferometers}{%
\section{Interferometers}\label{interferometers}}

\hypertarget{qubits-gates-and-circuits}{%
\section{Qubits, gates, and circuits}\label{qubits-gates-and-circuits}}

\hypertarget{quantum-decoherence}{%
\section{Quantum decoherence}\label{quantum-decoherence}}

\hypertarget{computation-deterministic-probabilistic-and-quantum}{%
\section{Computation: deterministic, probabilistic, and quantum}\label{computation-deterministic-probabilistic-and-quantum}}

\hypertarget{computational-complexity}{%
\section{Computational complexity}\label{computational-complexity}}

\hypertarget{outlook}{%
\section{Outlook}\label{outlook}}

\hypertarget{notes-and-exercises}{%
\section{Notes and Exercises}\label{notes-and-exercises}}

\hypertarget{supplement-physics-against-logic-via-beamsplitters}{%
\section{Supplement: Physics against logic, via beamsplitters}\label{supplement-physics-against-logic-via-beamsplitters}}

\hypertarget{supplement-quantum-interference-revisited-still-about-beamsplitters}{%
\section{Supplement: Quantum interference revisited (still about beamsplitters)}\label{supplement-quantum-interference-revisited-still-about-beamsplitters}}

\hypertarget{qubits}{%
\chapter{Qubits}\label{qubits}}

\hypertarget{measurements}{%
\chapter{Measurements}\label{measurements}}

\hypertarget{quantum-entanglement}{%
\chapter{Quantum entanglement}\label{quantum-entanglement}}

\hypertarget{quantum-algorithms}{%
\chapter{Quantum algorithms}\label{quantum-algorithms}}

\hypertarget{bells-theorem}{%
\chapter{Bell's theorem}\label{bells-theorem}}

\hypertarget{decoherence-and-elements-of-quantum-error-correction}{%
\chapter{Decoherence, and elements of quantum error correction}\label{decoherence-and-elements-of-quantum-error-correction}}

\hypertarget{density-matrices}{%
\chapter{Density matrices}\label{density-matrices}}

\hypertarget{quantum-channels-or-cp-maps}{%
\chapter{Quantum channels (or CP maps)}\label{quantum-channels-or-cp-maps}}

\hypertarget{quantum-error-correction-and-fault-tolerance}{%
\chapter{Quantum error correction and fault tolerance}\label{quantum-error-correction-and-fault-tolerance}}

  \bibliography{book.bib}

\end{document}
